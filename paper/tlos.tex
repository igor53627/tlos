\documentclass[10pt,a4paper,twocolumn]{article}
\usepackage[T1]{fontenc}
\usepackage{amsmath,amssymb,amsthm}
\usepackage{graphicx}
\usepackage{booktabs}
\usepackage{geometry}
\usepackage{listings}
\usepackage{xcolor}
\usepackage[colorlinks=true,linkcolor=blue,citecolor=blue,urlcolor=blue]{hyperref}
\geometry{a4paper,top=25mm,bottom=25mm,left=19mm,right=19mm,columnsep=8mm}

% TLO/TLOS notation macros
% TLO = Topology-Lattice Obfuscation
% TLOS = TLO + SEH (Subspace-Evasive Hashing from Ma-Dai-Shi 2025)
% 3-layer model: Topology (structural) + Lattice (C&C/LWE) + SEH (wire binding)

\newcommand{\tlo}{\textsc{TLO}}
\newcommand{\tlos}{\textsc{TLOS}}
\newcommand{\seh}{\ensuremath{\mathsf{SEH}}}
\newcommand{\xorgate}{\oplus}
\newcommand{\enc}{\mathsf{Enc}}
\newcommand{\dec}{\mathsf{Dec}}
\newcommand{\eval}{\mathsf{Eval}}
\newcommand{\obf}{\mathcal{O}}
\newcommand{\adv}{\mathcal{A}}
\newcommand{\negl}{\mathsf{negl}}
\newcommand{\ppt}{\ensuremath{\mathsf{PPT}}}
\newcommand{\indcpa}{\ensuremath{\mathsf{IND\text{-}CPA}}}
\newcommand{\wirestate}{\mathsf{WireState}}
\newcommand{\circuit}{\mathsf{Circuit}}
\newcommand{\gate}{\mathsf{Gate}}
\newcommand{\cac}{\ensuremath{\mathsf{C\&C}}}
\newcommand{\lwe}{\ensuremath{\mathsf{LWE}}}


\theoremstyle{plain}
\newtheorem{theorem}{Theorem}[section]
\newtheorem{proposition}[theorem]{Proposition}
\newtheorem{lemma}[theorem]{Lemma}
\newtheorem{corollary}[theorem]{Corollary}

\theoremstyle{definition}
\newtheorem{definition}[theorem]{Definition}

\lstset{
  basicstyle=\ttfamily\scriptsize,
  keywordstyle=\color{blue}\bfseries,
  commentstyle=\color{gray}\itshape,
  breaklines=true,
  frame=single,
  columns=fullflexible,
  upquote=true,
}

\title{\tlos{}: Topology-Lattice Obfuscation for Smart Contracts}
\author{
\textit{Anonymous Submission}
}
\date{}

\begin{document}
\maketitle

\begin{abstract}
We present \tlos{}, a practical circuit obfuscation framework for EVM smart contracts that resists specific attack classes (structural, semantic, mix-and-match) with practical gas costs.

\tlos{} uses a three-layer security model: (1)~topology mixing complicates structural attacks via non-power-of-2 wire distances and uniform wire usage, (2)~noiseless LWE-like (LBLO) inner products computed on-chain hide gate control functions, and (3)~a full-rank linear hash (inspired by~\cite{MDS25}) binds wire values across gates, ensuring that an adversary cannot mix outputs from different execution paths.

\tlos{} achieves ${\sim}8.5$M gas with $n{=}128$ LBLO (LWE-like) dimension and full-rank $64{\times}64$ wire binding. Security estimates (${\sim}2^{98}$ PQ) are based on heuristic modeling of our noiseless LBLO construction; see \S\ref{sec:lwe-params} for full parameterization. Target applications include predicates with eventually-expiring secrets, particularly \emph{low-entropy secrets} (passwords, phrases, small ranges) where \tlos{} provides significant advantage over simple hash commitments by making each guess computationally expensive.
\end{abstract}

%------------------------------------------------------------------------------
\section{Introduction}
%------------------------------------------------------------------------------

Smart contracts are fully transparent. This conflicts with applications requiring hidden logic: honeypots, MEV-resistant execution, sealed-bid auctions, and private thresholds.

Indistinguishability obfuscation (iO) from lattices~\cite{MDS25} achieves strong security but with impractical overhead. We take a different approach: \emph{practical obfuscation} that resists known attack classes through three complementary layers:

\begin{center}
\framebox{\parbox{0.85\columnwidth}{
\textbf{Three-Layer Security Model:}
\begin{enumerate}
\item \textbf{Topology:} Structural mixing complicates structural/statistical attacks. \emph{Security: heuristic.}
\item \textbf{LBLO (LWE-like, noiseless):} On-chain inner products hide control functions. \emph{Security: heuristic estimate ${\sim}2^{98}$ PQ for our LBLO parameters; not standard LWE (see \S\ref{sec:lwe-params}).}
\item \textbf{Wire Binding:} Full-rank linear hash binds wire values across gates, preventing mix-and-match. \emph{Security: algebraic binding (bijective map).}
\end{enumerate}
}}
\end{center}

\subsection{Contributions}

\begin{enumerate}
\item \textbf{\tlos{} Framework:} Three-layer obfuscation for EVM using lattice assumptions (${\sim}8.5$M gas with $n{=}128$).
\item \textbf{Full-Rank Wire Binding:} Inter-gate wire binding via on-chain $64{\times}64$ matrix-vector products. Inspired by~\cite{MDS25} but simplified: a public bijective map providing algebraic binding, not cryptographic hiding.
\item \textbf{Per-Batch Updates:} Gas-efficient binding (128-gate batches, 5 updates for 640 gates).
\item \textbf{PRG Optimization:} Derive 16 matrix coefficients per keccak call (320 calls vs 4096).
\item \textbf{Use Case Analysis:} When circuit obfuscation provides advantage over simple hash commitments.
\end{enumerate}

%------------------------------------------------------------------------------
\section{Advantage Over Hash Commitments}
\label{sec:advantage}
%------------------------------------------------------------------------------

A natural question: why not simply use \texttt{keccak256(secret)}? For random 256-bit secrets, keccak is indeed simpler. However, \tlos{} excels for \emph{low-entropy secrets} and \emph{multi-bit payloads}.

\begin{table}[h]
\centering
\scriptsize
\begin{tabular}{@{}lll@{}}
\toprule
\textbf{Secret Type} & \textbf{Keccak} & \textbf{\tlos{}} \\
\midrule
Random 256-bit & $2^{256}$ hashes & $\min(2^{256}, 2^{98})$\textsuperscript{$\dagger$} \\
Human password (10\textsuperscript{6}) & $<$1 ms & $\sim$17 min \\
Range 0--100K & $<$0.1 ms & $\sim$2.8 hours \\
4-word phrase (2048\textsuperscript{4}) & $\sim$18 sec & $\sim$5 years \\
\bottomrule
\end{tabular}
\caption{Offline brute-force cost comparison. \textsuperscript{$\dagger$}For high-entropy secrets, \tlos{} security is bounded by the conjectured hardness of our LBLO construction (heuristic estimate ${\sim}2^{98}$ with $n{=}128$; see \S\ref{sec:lwe-params}). Assumptions: keccak at $10^9$ hashes/s, \tlos{} at 100\,ms/guess.}
\label{tab:comparison}
\end{table}

\subsection{Deployment Constraints}

\tlos{} is designed for environments where:
\begin{itemize}
\item \textbf{On-chain only:} No off-chain computation or trusted servers
\item \textbf{No zk-SNARKs:} Proof systems unavailable or too complex
\item \textbf{No private L2:} Must deploy on public L1
\item \textbf{Gas is cheap:} At 0.03 gwei, 8.5M gas $\approx$ \$0.80
\end{itemize}

Under these constraints, \tlos{} provides a practical way to make low-entropy secret verification expensive on EVM---no memory-hard KDF (Argon2/scrypt) exists as an EVM precompile.

\subsection{Why Not Keccak?}

\textbf{Low-entropy secrets.} Keccak can be brute-forced at ${\sim}1$ns/guess. \tlos{} transforms each guess into an expensive LBLO-style (noiseless LWE-like) inner-product computation (${\sim}100$ms/guess), a $10^8\times$ slowdown.

\textbf{Multi-bit payloads.} With keccak, a payload requires separate ciphertext $\mathsf{Enc}_k(\mathsf{payload})$ alongside $\mathsf{keccak256}(k)$. Attackers who crack $k$ decrypt offline. In \tlos{}, payload is \emph{fused} into the circuit.

\textbf{Hidden structure.} Keccak reveals what you're checking (``it's a hash comparison''). \tlos{} hides the check structure---attackers don't know if it's one condition or ten.

\subsection{Practical Use Cases}

\textbf{DeFi:}
\begin{itemize}
\item \emph{Hidden liquidation thresholds:} MEV bots can't front-run if they don't know your trigger price
\item \emph{Private stop-loss/take-profit:} Strategy parameters hidden inside circuit
\item \emph{Dark limit orders:} Price bands obfuscated from sandwich attacks
\end{itemize}

\textbf{Wallet Recovery \& Inheritance:}
\begin{itemize}
\item \emph{Human-memorable recovery codes:} 6-word phrase protects wallet; keccak would be instant to crack, \tlos{} takes hours
\item \emph{On-chain wills:} Family passphrase unlocks inheritance; structure (who gets what) is hidden
\item \emph{Dead man's switch:} Liveness check + secret claim code, both obfuscated
\end{itemize}

\textbf{Gaming \& NFTs:}
\begin{itemize}
\item \emph{On-chain puzzles:} Low-entropy answers protected from GPU brute-force
\item \emph{Sealed-bid auctions:} Bid values from small domain can't be dictionary-attacked
\item \emph{Mystery box traits:} Generation logic hidden; can't predict rare mints
\end{itemize}

\textbf{When to use keccak:} Random 256-bit secrets with no payload and no structure to hide.

%------------------------------------------------------------------------------
\section{Construction}
%------------------------------------------------------------------------------

\subsection{Full-Rank Wire Binding}

We bind wire values across gates using a \emph{full-rank} linear hash: $H(x) = Ax \mod q$ where $A \in \mathbb{Z}_q^{64 \times 64}$ is derived from \texttt{circuitSeed} and gate index. This is inspired by~\cite{MDS25} but is \emph{not} subspace-evasive in the formal sense---it is simply a public bijective linear map providing algebraic binding.

\textbf{Why 64$\times$64?} A full-rank $64 \times 64$ matrix over $\mathbb{Z}_q$ has trivial kernel (only the zero vector), so $H$ is a bijection on $\mathbb{Z}_q^{64}$: every accumulator state has a unique preimage. Smaller matrices (e.g., $8 \times 64$) have large nullspaces where many inputs collide, undermining binding.

\textbf{Threat model (mix-and-match):} Without wire binding, an adversary could evaluate different branches of the circuit and splice together wire values from distinct execution paths to synthesize a ``fake'' trace. The full-rank hash forces all wires in a batch to be combined into a 1024-bit accumulator; any deviation changes the accumulator in a way that cannot be repaired without recomputing a full, consistent trace.

\textbf{Note:} This is a public random linear map over $\mathbb{Z}_q$ (no noise, no secret). Security is \emph{integrity/binding}, not cryptographic collision resistance: once an accumulator value is fixed, there is (for full-rank $A$) a single consistent wire vector that can produce it.

\begin{lstlisting}
// Optimized: 16 coefficients per keccak (320 calls vs 4096)
function _wireBindingHash(uint256 input, uint256 gateIdx) 
    internal view returns (uint256[4] memory output) {
  assembly {
    for { let row := 0 } lt(row, 64) { row := add(row, 1) } {
      let rowSeed := keccak256(seed, gateIdx, row)  // 1 per row
      let sum := 0
      for { let blockIdx := 0 } lt(col, nWires) { ... } {
        let blockDigest := keccak256(rowSeed, blockIdx)  // 16 coefs
        // Extract 16 u16 coefficients from blockDigest
        for { let k := 0 } lt(k, 16) { k := add(k, 1) } {
          let aij := mod(and(shr(mul(k,16), blockDigest), 0xFFFF), q)
          if bitVal { sum := add(sum, aij) }
        }
      }
      // Pack into output
    }
  }
}
\end{lstlisting}

\subsection{Per-Batch Updates}

For gas efficiency, the consistency hash is updated every 128 gates:
\[
\mathsf{hashAcc}_{k+1} = H(\mathsf{hashAcc}_k \oplus \mathsf{wires}, \mathsf{batchEnd}_k)
\]

This provides 5 hash updates for 640 gates. Combined with the PRG optimization (16 coefficients per keccak instead of 1), total wire binding overhead is reduced by ${\sim}25\times$ compared to naive implementation.

\subsection{LWE-Like Parameterization}
\label{sec:lwe-params}

Our control-function hiding uses an LWE-like construction with the following parameters:

\begin{center}
\begin{tabular}{@{}ll@{}}
\toprule
\textbf{Parameter} & \textbf{Value} \\
\midrule
Dimension $n$ & 128 \\
Modulus $q$ & 65521 (largest 16-bit prime) \\
Sample count $m$ & 2560 (4 ciphertexts $\times$ 640 gates) \\
Secret distribution & Uniform over $\mathbb{Z}_q^n$ (Keccak-derived) \\
Public vector $a$ & Uniform over $\mathbb{Z}_q^n$ \\
Error term $e$ & \textbf{None} (see below) \\
\bottomrule
\end{tabular}
\end{center}

\textbf{Encoding (no noise):} Encryption is deterministic: $b = \langle a, s \rangle + \mu \cdot (q/2) \mod q$, where $\mu \in \{0,1\}$ encodes the control-function bit. Decryption uses threshold $q/4$: if $(b - \langle a,s \rangle) \mod q$ falls in $(q/4, 3q/4)$, output 1; else 0.

\textbf{Why no error term?} EVM requires deterministic execution---probabilistic decoding failures would break on-chain verification. The tradeoff: we get \emph{perfect correctness} (0\% decoding error) but lose standard \lwe{} semantic security guarantees. Security relies on the difficulty of recovering $s$ without knowing the $\mu$ bits; subset attacks require $2^n$ guesses (see below).

\textbf{Security estimate (heuristic):} Our implementation does \emph{not} add an independent error term, so this is not standard LWE. The ${\sim}2^{98}$ PQ estimate is based on heuristic modeling as LWE$(n{=}128, q{=}65521, m{=}2560)$ using the lattice estimator~\cite{APS15}. \textbf{This should be interpreted as a rough complexity yardstick, not a formal security guarantee.} Standard LWE security reductions (to GapSVP, SIVP) do not apply. Security relies on the empirical observation that the best known attack (subset guessing) requires $2^n$ work.

\textbf{Binary-Offset Hardness:} Our construction is ``Learning with Binary Large Offset'' (LBLO): $b_i = \langle a_i, s \rangle + \mu_i \cdot (q/2) \mod q$ where $\mu_i \in \{0,1\}$. We implemented and tested 9 attack classes (\texttt{scripts/lblo\_attack.py}):

\begin{center}
\scriptsize
\begin{tabular}{@{}lll@{}}
\toprule
\textbf{Attack} & \textbf{Result} & \textbf{Notes} \\
\midrule
Naive brute-force & FAIL & $2^{2560}$ (all $\mu$)---infeasible \\
Statistical distinguishing & FAIL & $b$ uniform, $\chi^2{<}100$ \\
Lattice BDD & FAIL & Error norm $1.17{\times}10^6 \gg$ heuristic $19$ \\
Linear algebra & FAIL & Requires knowing $\mu$ \\
Subset guessing & FAIL & Still $2^n$ per subset \\
Hybrid (guess $k$ bits) & FAIL & Need $2^n$ guesses minimum \\
Meet-in-middle & FAIL & $s$ shared, not applicable \\
Actual BKZ-50 & FAIL & Finds only trivial vectors \\
$q/2$ structure & FAIL & No exploitable correlation \\
\bottomrule
\end{tabular}
\end{center}

\noindent\textbf{Exhaustive attack validation:} We ran multi-hour exhaustive searches on two servers to measure attack scaling:

\begin{center}
\scriptsize
\begin{tabular}{@{}lccc@{}}
\toprule
$n$ & Rate (ops/s) & Full $2^n$ time & Extrapolated $n{=}128$ \\
\midrule
16 & 1,500 & 44s & N/A \\
20 & 1,100 & 16min & N/A \\
24 & 820 & 5.7h & N/A \\
128 & N/A & N/A & $10^{28}$ years \\
\bottomrule
\end{tabular}
\end{center}

\noindent Attack parallelizes trivially, but even with $10^9$ machines: $10^{19}$ years.

\noindent\textbf{Caveat:} Absence of known attacks $\neq$ proven security. The ${\sim}2^{98}$ estimate is a \emph{heuristic} lower bound based on the best known attacks, not a formal security guarantee. Standard LWE reductions do not apply to our noiseless construction. We encourage independent cryptanalysis of the LBLO problem; our attack suite is available at \texttt{scripts/lblo\_attack.py}.

%------------------------------------------------------------------------------
\section{Security}
\label{sec:security}
%------------------------------------------------------------------------------

\begin{table}[h]
\centering
\scriptsize
\begin{tabular}{@{}lll@{}}
\toprule
\textbf{Component} & \textbf{Basis} & \textbf{Est.\ Security} \\
\midrule
CF hiding & LBLO (noiseless LWE-like, $n{=}128$) & ${\sim}2^{98}$ PQ (heuristic) \\
Wire binding & Full-rank 64$\times$64 linear map & Algebraic binding \\
Topology & Heuristic & Empirical (\S\ref{sec:topology-exp}) \\
\bottomrule
\end{tabular}
\caption{Security profile. LWE estimate is heuristic (see \S\ref{sec:lwe-params}). Wire binding provides algebraic binding via trivial kernel.}
\label{tab:security}
\end{table}

We estimate ${\sim}2^{98}$ post-quantum cost against the underlying LBLO instances by heuristically modeling them as LWE$(n{=}128, q{=}65521, m{=}2560)$ in standard lattice estimators. This should be interpreted as a rough complexity \emph{yardstick}, not a formal LWE-based security guarantee (see \S\ref{sec:lwe-params}). This security level is suitable for:
\begin{itemize}
\item Medium-term secrets (weeks--months)
\item Production applications with moderate value
\item Real-world deployment scenarios
\end{itemize}

For higher security (${\sim}2^{128}$ PQ), $n \geq 256$ would be needed (requires a high-throughput L2 or dedicated precompiles).

\subsection{What Wire Binding Provides}

\begin{itemize}
\item \textbf{Mix-and-match detection:} Any attempt to evaluate gates with inconsistent inputs will, except with negligible probability, yield a final accumulator that does not match the expected value.
\item \textbf{Execution trace binding:} Full evaluation history is committed.
\item \textbf{Lattice-style construction:} Uses the same linear algebra as the LBLO layer for design uniformity; its security role is \emph{integrity/binding}: for a fixed full-rank $A$, each accumulator state has a unique preimage in $\mathbb{Z}_q^{64}$ under $H$.
\end{itemize}

\subsection{Limitations}

\tlos{} does \emph{not} provide:
\begin{itemize}
\item Semantic security against black-box evaluation (not targeted; we resist specific attack classes only)
\item Full iO (indistinguishability obfuscation) (not targeted)
\end{itemize}

%------------------------------------------------------------------------------
\section{Implementation}
%------------------------------------------------------------------------------

\textbf{Contracts:} \texttt{TLOSLWE.sol} (${\sim}$350 lines Solidity)

\textbf{Rust generator:} Produces circuit data matching Solidity evaluation exactly. Outputs 662,400 bytes for 640 gates with $n{=}128$.

\textbf{Gas costs (measured):}
\begin{itemize}
\item \textbf{TLOS-LWE ($n{=}128$, full-rank wire binding)}: ${\sim}8.5$M gas (28\% of block)
\end{itemize}

\textbf{Optimizations applied:}
\begin{itemize}
\item PRG optimization: 16 coefficients per keccak (320 calls vs 4096 per update)
\item Single \texttt{mod} at end of inner product (vs per-term)
\item Batch size 128 (5 binding updates for 640 gates)
\item $n{=}128$ LWE dimension for ${\sim}2^{98}$ PQ security
\end{itemize}

%------------------------------------------------------------------------------
\section{Topology Mixing Experiments}
\label{sec:topology-exp}
%------------------------------------------------------------------------------

We empirically evaluated topology mixing against simple structural attacks on 100 random 64-wire/640-gate circuits.

\textbf{Attacker model:} Adversary sees the full obfuscated circuit graph (wire indices, gate structure) but not the LWE-encrypted control functions. Goal: identify semantic structure (e.g., comparators, adders) or reduce circuit complexity.

\textbf{Attacks tested:}
\begin{itemize}
\item \textbf{Degree analysis:} Identify high-fanout ``hub'' wires that might carry semantic signals.
\item \textbf{Pattern matching:} Search for known subcircuit patterns (XOR chains, comparator structures).
\item \textbf{Wire distance heuristic:} Exploit power-of-2 distances between control wires.
\end{itemize}

\textbf{Results:}
\begin{center}
\begin{tabular}{@{}lcc@{}}
\toprule
\textbf{Attack} & \textbf{Baseline} & \textbf{With Mixing} \\
\midrule
Hub wire identification & 78\% & 12\% \\
Pattern match (XOR chain) & 91\% & 8\% \\
Power-of-2 distance exploit & 100\% & 0\% \\
\bottomrule
\end{tabular}
\end{center}

Topology mixing uses non-power-of-2 wire distances and uniform wire usage, which eliminates the power-of-2 distance exploit entirely and significantly reduces hub/pattern detection rates. These results are empirical; we do not claim formal security.

\subsection{Layer Ablation}

We evaluated attack resistance with different layer combinations:

\begin{center}
\begin{tabular}{@{}lccc@{}}
\toprule
\textbf{Configuration} & \textbf{Structural} & \textbf{Semantic} & \textbf{Mix-Match} \\
\midrule
Baseline (no obfuscation) & 0/3 & 0/1 & 0/1 \\
Topology only & 3/3 & 0/1 & 0/1 \\
Topology + LBLO & 3/3 & 1/1 & 0/1 \\
Topology + LBLO + Wire Binding & 3/3 & 1/1 & 1/1 \\
\bottomrule
\end{tabular}
\end{center}

\textbf{Key findings:} (1)~Topology alone defeats structural attacks but not semantic (RainbowTable) attacks. (2)~LBLO is necessary to hide control functions from semantic analysis. (3)~Wire binding is necessary to prevent mix-and-match---without it, an attacker can evaluate gates with inconsistent wire values.

%------------------------------------------------------------------------------
\section{Conclusion}
%------------------------------------------------------------------------------

\tlos{} is a practical circuit obfuscation framework for EVM using lattice-inspired assumptions, achieving estimated ${\sim}2^{98}$ post-quantum security (heuristic, see \S\ref{sec:lwe-params}) at ${\sim}8.5$M gas with $n{=}128$ and full-rank wire binding. The three-layer design (topology + LBLO (LWE-like) control function hiding + full-rank wire binding) provides resistance against structural, statistical, semantic, and mix-and-match attacks.

Key insight: \tlos{} provides significant advantage over simple hash commitments for \emph{low-entropy secrets} by making each guess computationally expensive. Its value is in making low-entropy guesses expensive under EVM constraints and enabling richer obfuscated logic.

Future work includes higher security parameters ($n \geq 256$), succinct verification via zkSNARK/STARK, and formal security analysis of batch wire binding.

\begin{thebibliography}{9}
\bibitem{MDS25} Ma, Dai, Shi. ``Indistinguishability Obfuscation from Lattices in the Quantum Random Oracle Model.'' ePrint 2025/307.
\bibitem{Regev05} Regev. ``On Lattices, Learning with Errors, Random Linear Codes, and Cryptography.'' STOC 2005.
\bibitem{GKW17} Goyal, Koppula, Waters. ``Lockable Obfuscation.'' FOCS 2017.
\bibitem{APS15} Albrecht, Player, Scott. ``On the Concrete Hardness of Learning with Errors.'' J.\ Math.\ Cryptol.\ 2015.
\end{thebibliography}

\end{document}
