\documentclass[10pt,a4paper,twocolumn]{article}
\usepackage[T1]{fontenc}
\usepackage{amsmath,amssymb,amsthm}
\usepackage{graphicx}
\usepackage{booktabs}
\usepackage{geometry}
\usepackage{listings}
\usepackage{xcolor}
\usepackage[colorlinks=true,linkcolor=blue,citecolor=blue,urlcolor=blue]{hyperref}
\geometry{a4paper,top=25mm,bottom=25mm,left=19mm,right=19mm,columnsep=8mm}

% TLO/TLOS notation macros
% TLO = Topology-Lattice Obfuscation
% TLOS = TLO + SEH (Subspace-Evasive Hashing from Ma-Dai-Shi 2025)
% 3-layer model: Topology (structural) + Lattice (C&C/LWE) + SEH (wire binding)

\newcommand{\tlo}{\textsc{TLO}}
\newcommand{\tlos}{\textsc{TLOS}}
\newcommand{\seh}{\ensuremath{\mathsf{SEH}}}
\newcommand{\xorgate}{\oplus}
\newcommand{\enc}{\mathsf{Enc}}
\newcommand{\dec}{\mathsf{Dec}}
\newcommand{\eval}{\mathsf{Eval}}
\newcommand{\obf}{\mathcal{O}}
\newcommand{\adv}{\mathcal{A}}
\newcommand{\negl}{\mathsf{negl}}
\newcommand{\ppt}{\ensuremath{\mathsf{PPT}}}
\newcommand{\indcpa}{\ensuremath{\mathsf{IND\text{-}CPA}}}
\newcommand{\wirestate}{\mathsf{WireState}}
\newcommand{\circuit}{\mathsf{Circuit}}
\newcommand{\gate}{\mathsf{Gate}}
\newcommand{\cac}{\ensuremath{\mathsf{C\&C}}}
\newcommand{\lwe}{\ensuremath{\mathsf{LWE}}}


\theoremstyle{plain}
\newtheorem{theorem}{Theorem}[section]
\newtheorem{proposition}[theorem]{Proposition}
\newtheorem{lemma}[theorem]{Lemma}
\newtheorem{corollary}[theorem]{Corollary}

\theoremstyle{definition}
\newtheorem{definition}[theorem]{Definition}

\lstset{
  basicstyle=\ttfamily\scriptsize,
  keywordstyle=\color{blue}\bfseries,
  commentstyle=\color{gray}\itshape,
  breaklines=true,
  frame=single,
  columns=fullflexible,
  upquote=true,
}

\title{\tlos{}: Topology-Lattice Obfuscation for Smart Contracts}
\author{
\textit{Anonymous Submission}
}
\date{}

\begin{document}
\maketitle

\begin{abstract}
How can we protect low-entropy secrets---like human-memorable passwords or MEV bot triggers---on a fully transparent blockchain? We propose \tlos{}, a lightweight obfuscation framework that secures sensitive predicates against GPU brute-force attacks. By layering standard LWE cryptography with a ``planted puzzle'' proof-of-work, \tlos{} forces attackers to spend significant computation per guess ($2^{76}$ search space), while legitimate verification remains cheap (1.8M--6M gas for $n{=}384$, depending on circuit size). This approach bridges the gap between theoretical obfuscation and practical deployment, enabling new classes of privacy-preserving applications on Ethereum Mainnet.
\end{abstract}

%------------------------------------------------------------------------------
\section{Introduction}
%------------------------------------------------------------------------------

Smart contracts are fully transparent. This conflicts with applications requiring hidden logic: honeypots, MEV-resistant execution, sealed-bid auctions, and private thresholds.

Indistinguishability obfuscation (iO) from lattices~\cite{MDS25} achieves strong security but with impractical overhead. We take a different approach: \emph{practical obfuscation} that resists known attack classes through four complementary layers:

\begin{center}
\framebox{\parbox{0.85\columnwidth}{
\textbf{Four-Layer Defense-in-Depth:}
\begin{enumerate}
\item \textbf{Topology (stops the reverse engineer):} Structural mixing raises the bar against pattern-matching. \emph{Security: heuristic (hard security from Layers 2 \& 4).}
\item \textbf{LWE (stops the cryptanalyst):} On-chain inner products hide control functions. \emph{Security: ${\sim}2^{112}$ PQ for $n{=}384$.}
\item \textbf{Wire Binding (stops the hacker):} Full-rank linear hash prevents mix-and-match. \emph{Security: algebraic binding.}
\item \textbf{Puzzle (stops the GPU miner):} Forces $3^{48} \approx 2^{76}$ brute-force work. \emph{Security: computational.}
\end{enumerate}
}}
\end{center}

\subsection{Contributions}

\begin{enumerate}
\item \textbf{\tlos{} Framework:} Four-layer obfuscation for EVM using lattice assumptions (1.8M--6M gas with $n{=}384$).
\item \textbf{Full-Rank Wire Binding:} Inter-gate wire binding via on-chain $64{\times}64$ matrix-vector products. Inspired by~\cite{MDS25} but simplified: a public bijective map providing algebraic binding, not cryptographic hiding.
\item \textbf{Per-Batch Updates:} Gas-efficient binding (128-gate batches, 5 updates for 640 gates).
\item \textbf{Layer 4 Puzzle:} Planted LWE puzzle ($n{=}48$, $m{=}72$, $q{=}2039$) forces $2^{76}$ minimum search space at 1.26M gas overhead.
\item \textbf{PRG Optimization:} Derive 16 matrix coefficients per keccak call (320 calls vs 4096).
\item \textbf{Use Case Analysis:} When circuit obfuscation provides advantage over simple hash commitments.
\end{enumerate}

%------------------------------------------------------------------------------
\section{Advantage Over Hash Commitments}
\label{sec:advantage}
%------------------------------------------------------------------------------

A natural question: why not simply use \texttt{keccak256(secret)}? For random 256-bit secrets, keccak is indeed simpler. However, \tlos{} excels for \emph{low-entropy secrets} and \emph{multi-bit payloads}.

\begin{table}[h]
\centering
\scriptsize
\begin{tabular}{@{}lll@{}}
\toprule
\textbf{Secret Type} & \textbf{Keccak} & \textbf{\tlos{}} \\
\midrule
Random 256-bit & $2^{256}$ hashes & $\min(2^{256}, 2^{112})$\textsuperscript{$\dagger$} \\
Human password (10\textsuperscript{6}) & $<$1 ms & $\sim$10 sec \\
Range 0--100K & $<$0.1 ms & $\sim$1 sec \\
4-word phrase (2048\textsuperscript{4}) & $\sim$18 sec & $\sim$5 months \\
\bottomrule
\end{tabular}
\caption{Offline brute-force cost (GPU). \textsuperscript{$\dagger$}For high-entropy secrets, \tlos{} security is bounded by LWE hardness (${\sim}2^{112}$). Assumptions: keccak at $10^9$ hashes/s (GPU), \tlos{} at $10^5$ guesses/s (GPU offline simulation).}
\label{tab:comparison}
\end{table}

\subsection{Deployment Constraints}

\tlos{} is designed for environments where:
\begin{itemize}
\item \textbf{On-chain only:} No off-chain computation or trusted servers
\item \textbf{No zk-SNARKs:} Proof systems unavailable or too complex
\item \textbf{No private L2:} Must deploy on public L1
\item \textbf{Gas is cheap:} At 0.03 gwei, 2M gas (128 gates) $\approx$ \$0.19
\end{itemize}

Under these constraints, \tlos{} provides a practical way to make low-entropy secret verification expensive on EVM---no memory-hard KDF (Argon2/scrypt) exists as an EVM precompile.

\subsection{Why Not Keccak?}

\textbf{Low-entropy secrets.} Keccak can be brute-forced at ${\sim}1$ns/guess ($10^9$/s on GPU). \tlos{} transforms each guess into an expensive LWE inner-product computation (${\sim}10\mu$s/guess, or $10^5$/s on GPU), a $10^4\times$ slowdown.

\textbf{Multi-bit payloads.} With keccak, a payload requires separate ciphertext $\mathsf{Enc}_k(\mathsf{payload})$ alongside $\mathsf{keccak256}(k)$. Attackers who crack $k$ decrypt offline. In \tlos{}, payload is \emph{fused} into the circuit.

\textbf{Hidden structure.} Keccak reveals what you're checking (``it's a hash comparison''). \tlos{} hides the check structure---attackers don't know if it's one condition or ten.

\subsection{Practical Use Cases}

\textbf{DeFi:}
\begin{itemize}
\item \emph{Hidden liquidation thresholds:} MEV bots can't front-run if they don't know your trigger price
\item \emph{Private stop-loss/take-profit:} Strategy parameters hidden inside circuit
\item \emph{Dark limit orders:} Price bands obfuscated from sandwich attacks
\end{itemize}

\textbf{Wallet Recovery \& Inheritance:}
\begin{itemize}
\item \emph{Human-memorable recovery codes:} 6-word phrase protects wallet; keccak would be instant to crack, \tlos{} takes hours
\item \emph{On-chain wills:} Family passphrase unlocks inheritance; structure (who gets what) is hidden
\item \emph{Dead man's switch:} Liveness check + secret claim code, both obfuscated
\end{itemize}

\textbf{Gaming \& NFTs:}
\begin{itemize}
\item \emph{On-chain puzzles:} Low-entropy answers protected from GPU brute-force
\item \emph{Sealed-bid auctions:} Bid values from small domain can't be dictionary-attacked
\item \emph{Mystery box traits:} Generation logic hidden; can't predict rare mints
\end{itemize}

\textbf{Identity \& Access Control:}
\begin{itemize}
\item \emph{ENS recovery:} Human-memorable codes protect high-value names
\item \emph{One-time delegation codes:} 6-digit codes; \tlos{} makes $10^6$ guesses take hours
\item \emph{Multi-code access:} OR of $N$ event codes with hidden structure
\end{itemize}

\textbf{When to use keccak:} Random 256-bit secrets with no payload and no structure to hide.

%------------------------------------------------------------------------------
\section{Construction}
%------------------------------------------------------------------------------

\subsection{Full-Rank Wire Binding}

We bind wire values across gates using a \emph{full-rank} linear hash: $H(x) = Ax \mod q$ where $A \in \mathbb{Z}_q^{64 \times 64}$ is derived from \texttt{circuitSeed} and gate index. This is inspired by~\cite{MDS25} but is \emph{not} subspace-evasive in the formal sense---it is simply a public bijective linear map providing algebraic binding.

\textbf{Why 64$\times$64?} A full-rank $64 \times 64$ matrix over $\mathbb{Z}_q$ has trivial kernel (only the zero vector), so $H$ is a bijection on $\mathbb{Z}_q^{64}$: every accumulator state has a unique preimage. The matrix is derived via PRG expansion from \texttt{circuitSeed}; for random $A$ over $\mathbb{Z}_q$, full-rank probability is $>1{-}10^{-19}$.

\textbf{Threat model (mix-and-match):} Without wire binding, an adversary could evaluate different branches of the circuit and splice together wire values from distinct execution paths to synthesize a ``fake'' trace. The full-rank hash forces all wires in a batch to be combined into a 1024-bit accumulator; any deviation changes the accumulator in a way that cannot be repaired without recomputing a full, consistent trace.

\textbf{Note:} This is a public random linear map over $\mathbb{Z}_q$ (no noise, no secret). Security is \emph{integrity/binding}, not cryptographic collision resistance: once an accumulator value is fixed, there is (for full-rank $A$) a single consistent wire vector that can produce it.

\begin{lstlisting}
// Optimized: 16 coefficients per keccak (320 calls vs 4096)
function _wireBindingHash(uint256 input, uint256 gateIdx) 
    internal view returns (uint256[4] memory output) {
  assembly {
    for { let row := 0 } lt(row, 64) { row := add(row, 1) } {
      let rowSeed := keccak256(seed, gateIdx, row)  // 1 per row
      let sum := 0
      for { let blockIdx := 0 } lt(col, nWires) { ... } {
        let blockDigest := keccak256(rowSeed, blockIdx)  // 16 coefs
        // Extract 16 u16 coefficients from blockDigest
        for { let k := 0 } lt(k, 16) { k := add(k, 1) } {
          let aij := mod(and(shr(mul(k,16), blockDigest), 0xFFFF), q)
          if bitVal { sum := add(sum, aij) }
        }
      }
      // Pack into output
    }
  }
}
\end{lstlisting}

\subsection{Per-Batch Updates}

For gas efficiency, the consistency hash is updated every 128 gates:
\[
\mathsf{hashAcc}_{k+1} = H(\mathsf{hashAcc}_k \oplus \mathsf{wires}, \mathsf{batchEnd}_k)
\]

This provides 5 hash updates for 640 gates. Combined with the PRG optimization (16 coefficients per keccak instead of 1), total wire binding overhead is reduced by ${\sim}25\times$ compared to naive implementation.

\subsection{Planted LWE Puzzle (Layer 4)}
\label{sec:puzzle}

Layers 1--3 protect against structural, semantic, and mix-and-match attacks, but cannot prevent GPU brute-force over low-entropy inputs. If the secret has only $2^{30}$ entropy, an attacker evaluating at $10^6$ guesses/sec cracks it in ${\sim}17$ minutes.

Layer 4 forces a minimum search space by requiring solvers to find a ternary secret $s \in \{-1,0,1\}^{48}$ satisfying:
\[
\|As - b\|^2 < \tau
\]
where $(A, b)$ are deterministically derived from the input $x$. The puzzle uses a \emph{planted} construction: for each $x$, we derive a planted secret $s^*$ and error $e$, then set $b = As^* + e \mod q$. The solver must recover $s^*$ (or an equivalent short solution).

\begin{center}
\begin{tabular}{@{}ll@{}}
\toprule
\textbf{Parameter} & \textbf{Value} \\
\midrule
Secret dimension $n$ & 48 \\
Samples $m$ & 72 \\
Modulus $q$ & 2039 \\
Error range & $\{-2,-1,0,1,2\}$ \\
Threshold $\tau$ & 300 \\
Search space & $3^{48} \approx 2^{76}$ \\
Verification gas & 1.26M \\
\bottomrule
\end{tabular}
\end{center}

\textbf{Security:} GPU brute-force at 436M guesses/sec (GH200) requires ${\sim}5.7$ million years for exhaustive search. Even 10,000 GPUs (${\sim}570$ years) cannot crack in practical time.

\textbf{Integration:} The puzzle solution hash $H(s)$ is combined with the input to derive the LWE secret, providing double binding: attackers must solve the puzzle AND have the correct input.

\subsection{LWE Parameterization}
\label{sec:lwe-params}

Our control-function hiding uses standard LWE with Gaussian noise:

\begin{center}
\begin{tabular}{@{}ll@{}}
\toprule
\textbf{Parameter} & \textbf{Value} \\
\midrule
Dimension $n$ & 384 \\
Modulus $q$ & 65521 (largest 16-bit prime) \\
Sample count $m$ & 2560 (4 ciphertexts $\times$ 640 gates) \\
Secret distribution & Uniform over $\mathbb{Z}_q^n$ (Keccak-derived) \\
Public vector $a$ & Seed-derived over $\mathbb{Z}_q^n$ (11 bytes/gate) \\
Error distribution & Gaussian with $\sigma{=}8$ \\
\bottomrule
\end{tabular}
\end{center}

\textbf{Encoding:} Encryption uses standard LWE: $b = \langle a, s \rangle + e + \mu \cdot (q/2) \mod q$, where $\mu \in \{0,1\}$ encodes the control-function bit and $e \sim \mathcal{N}(0, \sigma^2)$ is Gaussian noise. Decryption uses threshold $q/4$: if $(b - \langle a,s \rangle) \mod q$ falls in $(q/4, 3q/4)$, output 1; else 0.

\textbf{Correctness:} With $\sigma{=}8$ and threshold at $q/4 \approx 16380$, decryption error probability is negligible ($< 2^{-40}$). The Gaussian noise provides semantic security while maintaining practical correctness.

\textbf{Security estimate:} Using the lattice estimator~\cite{APS15} for LWE$(n{=}384, q{=}65521, \sigma{=}8, m{=}2560)$, we obtain ${\sim}2^{112}$ post-quantum security. This follows standard LWE hardness assumptions with reductions to GapSVP/SIVP.

\textbf{LWE Hardness:} Our construction uses standard LWE: $b_i = \langle a_i, s \rangle + e_i + \mu_i \cdot (q/2) \mod q$ where $e_i \sim \mathcal{N}(0, \sigma^2)$ with $\sigma{=}8$. The ${\sim}2^{112}$ post-quantum estimate is based on the lattice estimator~\cite{APS15} using the dual attack with BKZ/sieving cost model. Worst-case to average-case reductions~\cite{Regev05} justify LWE as a hardness assumption, but our concrete security figure is based on best-known algorithms, not the reductions themselves.

\textbf{Validation:} We implemented 9 attack classes (\texttt{scripts/tlos\_attack.py}) to verify correctness and confirm no obvious breaks exist. These exploratory tests are \emph{not} a substitute for the estimator-based security analysis; they only confirm the implementation is sound.

%------------------------------------------------------------------------------
\section{Security}
\label{sec:security}
%------------------------------------------------------------------------------

\begin{table}[h]
\centering
\scriptsize
\begin{tabular}{@{}lp{2.2cm}l@{}}
\toprule
\textbf{Layer} & \textbf{Basis} & \textbf{Security} \\
\midrule
CF hiding & LWE ($n{=}384$, $\sigma{=}8$) & ${\sim}2^{112}$ \\
Wire binding & 64$\times$64 linear & Algebraic \\
Topology & Heuristic & Empirical \\
Puzzle & Ternary LWE ($n{=}48$) & $2^{76}$ min \\
\bottomrule
\end{tabular}
\caption{Security profile. LWE with Gaussian noise ($\sigma{=}8$).}
\label{tab:security}
\end{table}

We estimate ${\sim}2^{112}$ post-quantum cost against the underlying LWE instances using the lattice estimator for LWE$(n{=}384, q{=}65521, \sigma{=}8, m{=}2560)$. This follows standard LWE hardness with reductions to GapSVP/SIVP (see \S\ref{sec:lwe-params}). This security level is suitable for:
\begin{itemize}
\item Medium-term secrets (weeks--months)
\item Production applications with moderate value
\item Real-world deployment scenarios
\end{itemize}

For even higher security margins, $n \geq 512$ could be considered with modest gas increase.

\subsection{Combined Security Model}

Security relies on \emph{both} layers working together:
\begin{itemize}
\item \textbf{Topology} hides $\mu$ (control function bits) from structural analysis
\item \textbf{LWE} hides $s$ given encrypted $\mu$ values
\item \textbf{Wire Binding} ensures execution trace consistency (integrity, not hiding)
\item \textbf{Puzzle} forces minimum work per guess ($2^{76}$)
\end{itemize}

\textbf{Composition rule:} For a point-function circuit on input space $\mathcal{X}$ with entropy $h$, the cost of finding a preimage is:
\[
\min\big(2^h,\ 2^{112},\ 2^{76}\big)
\]
where $2^h$ is brute-force over inputs, $2^{112}$ is LWE security, and $2^{76}$ is the puzzle's per-guess cost floor. Layers 1 and 3 raise the bar against structural/mix-and-match \emph{shortcuts} but do not multiply the security level.

The ${\sim}2^{112}$ estimate assumes $\mu$ cannot be predicted from circuit structure. If an attacker predicts $\mu$ for $n{=}384$ gates via structural analysis, they can recover $s$ via Gaussian elimination in $O(n^3)$.

\textbf{Why structural attacks fail:} TLOS circuits compute point functions ($C(x){=}1$ iff $x{=}\mathsf{secret}$). There are no adders, comparators, or recognizable gadgets---only reversible permutations with one distinguished output.

\textbf{Input-dependent key:} The decryption key $s{=}H(\mathsf{input})$ is derived at evaluation time. Tracing with wrong input $x'$ yields random $\mathsf{diff}$ values, revealing nothing about $\mu$. \emph{Traces are visible but useless without the correct input.}

\subsection{What Wire Binding Provides}

\begin{itemize}
\item \textbf{Mix-and-match detection:} Any attempt to evaluate gates with inconsistent inputs will, except with negligible probability, yield a final accumulator that does not match the expected value.
\item \textbf{Execution trace binding:} Full evaluation history is committed.
\item \textbf{Lattice-style construction:} Uses the same linear algebra as the LWE layer for design uniformity; its security role is \emph{integrity/binding}: for a fixed full-rank $A$, each accumulator state has a unique preimage in $\mathbb{Z}_q^{64}$ under $H$.
\end{itemize}

\subsection{Limitations}

\tlos{} does \emph{not} provide:
\begin{itemize}
\item Semantic security against black-box evaluation (not targeted; we resist specific attack classes only)
\item Full iO (indistinguishability obfuscation) (not targeted)
\end{itemize}

%------------------------------------------------------------------------------
\section{Implementation}
%------------------------------------------------------------------------------

\textbf{Contracts:} \texttt{TLOSLWE.sol} (${\sim}$350 lines Solidity)

\textbf{Rust generator:} Produces circuit data matching Solidity evaluation exactly. Storage is 11 bytes/gate via seed-derived $a$ vectors (vs.\ explicit storage).

\textbf{Gas costs (measured, $n{=}384$, LWE + wire binding only):}
\begin{itemize}
\item \textbf{64 gates}: ${\sim}$1.8M gas (3\% of block)
\item \textbf{128 gates}: ${\sim}$2M gas (3\% of block)
\item \textbf{256 gates}: ${\sim}$3M gas (5\% of block)
\end{itemize}
\textbf{Full 4-layer (add puzzle overhead):} +1.26M gas for Layer 4 puzzle verification. Example: 128 gates full = ${\sim}$3.3M gas (5.5\% of block).

\textbf{Optimizations applied:}
\begin{itemize}
\item PRG optimization: 16 coefficients per keccak (320 calls vs 4096 per update)
\item Single \texttt{mod} at end of inner product (vs per-term)
\item Batch size 128 (5 binding updates for 640 gates)
\item $n{=}384$ LWE dimension for ${\sim}2^{112}$ PQ security
\end{itemize}

%------------------------------------------------------------------------------
\section{Topology Mixing Experiments}
\label{sec:topology-exp}
%------------------------------------------------------------------------------

We empirically evaluated topology mixing against simple structural attacks on 100 random 64-wire/640-gate circuits.

\textbf{Attacker model:} Adversary sees the full obfuscated circuit graph (wire indices, gate structure) but not the LWE-encrypted control functions. Goal: identify semantic structure (e.g., comparators, adders) or reduce circuit complexity.

\textbf{Attacks tested:}
\begin{itemize}
\item \textbf{Degree analysis:} Identify high-fanout ``hub'' wires that might carry semantic signals.
\item \textbf{Pattern matching:} Search for known subcircuit patterns (XOR chains, comparator structures).
\item \textbf{Wire distance heuristic:} Exploit power-of-2 distances between control wires.
\end{itemize}

\textbf{Results:}
\begin{center}
\begin{tabular}{@{}lcc@{}}
\toprule
\textbf{Attack} & \textbf{Baseline} & \textbf{With Mixing} \\
\midrule
Hub wire identification & 78\% & 12\% \\
Pattern match (XOR chain) & 91\% & 8\% \\
Power-of-2 distance exploit & 100\% & 0\% \\
\bottomrule
\end{tabular}
\end{center}

Topology mixing uses non-power-of-2 wire distances and uniform wire usage, which eliminates the power-of-2 distance exploit entirely and significantly reduces hub/pattern detection rates. These results are empirical; we do not claim formal security.

\subsection{Layer Ablation}

We evaluated attack resistance with different layer combinations:

\begin{center}
\scriptsize
\begin{tabular}{@{}lcccc@{}}
\toprule
\textbf{Configuration} & \textbf{Struct.} & \textbf{Semantic} & \textbf{Mix-Match} & \textbf{Brute-Force} \\
\midrule
Baseline (no obfuscation) & 0/3 & 0/1 & 0/1 & 0/1 \\
Topology only & 3/3 & 0/1 & 0/1 & 0/1 \\
Topology + LWE & 3/3 & 1/1 & 0/1 & 0/1 \\
Topology + LWE + Binding & 3/3 & 1/1 & 1/1 & 0/1 \\
Full 4-layer (+ Puzzle) & 3/3 & 1/1 & 1/1 & 1/1 \\
\bottomrule
\end{tabular}
\end{center}

\textbf{Key findings:} (1)~Topology alone defeats structural attacks but not semantic (RainbowTable) attacks. (2)~LWE is necessary to hide control functions from semantic analysis. (3)~Wire binding is necessary to prevent mix-and-match---without it, an attacker can evaluate gates with inconsistent wire values. (4)~The planted LWE puzzle forces minimum $2^{76}$ brute-force work regardless of input entropy, protecting low-entropy secrets from GPU attacks.

%------------------------------------------------------------------------------
\section{Conclusion}
%------------------------------------------------------------------------------

\tlos{} is a practical circuit obfuscation framework for EVM using lattice-based assumptions, achieving ${\sim}2^{112}$ post-quantum security at 1.8M--6M gas with $n{=}384$ and full-rank wire binding. The four-layer design (topology + LWE control function hiding + full-rank wire binding + planted LWE puzzle) provides resistance against structural, statistical, semantic, mix-and-match, and brute-force attacks.

Key insight: \tlos{} provides significant advantage over simple hash commitments for \emph{low-entropy secrets} by making each guess computationally expensive. Its value is in making low-entropy guesses expensive under EVM constraints and enabling richer obfuscated logic.

Future work includes even higher security parameters ($n \geq 1024$), succinct verification via zkSNARK/STARK, and formal security analysis of batch wire binding.

\begin{thebibliography}{9}
\bibitem{MDS25} Ma, Dai, Shi. ``Indistinguishability Obfuscation from Lattices in the Quantum Random Oracle Model.'' ePrint 2025/307.
\bibitem{Regev05} Regev. ``On Lattices, Learning with Errors, Random Linear Codes, and Cryptography.'' STOC 2005.
\bibitem{GKW17} Goyal, Koppula, Waters. ``Lockable Obfuscation.'' FOCS 2017.
\bibitem{APS15} Albrecht, Player, Scott. ``On the Concrete Hardness of Learning with Errors.'' J.\ Math.\ Cryptol.\ 2015.
\end{thebibliography}

\end{document}
